%% LyX 2.0.1 created this file.  For more info, see http://www.lyx.org/.
%% Do not edit unless you really know what you are doing.
\documentclass[english]{article}
\usepackage[T1]{fontenc}
\usepackage[latin9]{inputenc}
\usepackage{babel}
\begin{document}

\title{Software Freedom Conservancy Conflict of Interest Policy}

\maketitle

\section{Purpose}

The purpose of this conflict of interest policy (``Policy'') is
to protect the Software Freedom Conservancy (``Conservancy'') and
its member Projects when Conservancy is contemplating entering into
a transaction or arrangement that might benefit the private interest
of a Director, Officer or Staff Member of Conservancy or a Representative
of a Project Leadership Committee (``PLC''), or might result in
a possible excess benefit transaction. This Policy is intended to
supplement but not replace any applicable state and federal laws governing
conflict of interest applicable to nonprofit and charitable organizations.


\section{Conservancy Directors, Officers and Staff}

Directors, Officers and Staff Members of Conservancy (``Conservancy
Interested Persons'') each have a duty to protect Conservancy and
its Member Projects from violating state and federal laws - and to
avoid any appearance of impropriety. Conservancy Interested Persons
serve the public interest and are to have a clear understanding of
Conservancy's charitable mission. All decisions made by Conservancy
Interested Persons are to be made solely on the basis of a desire
to promote the best interests of Conservancy and the public good. 


\subsection{Defining a Conflict of Interest for a Conservancy Interested Person}

In general, Conservancy Interested Persons should avoid making decisions
on matters where their personal interests are at odds with the Conservancy's
interests. In particular, the following scenarios are to be identified
as conflicts of interest:
\begin{itemize}
\item A Conservancy Interested Person (or a family member of same) is a
party to a contract, or involved in a transaction with Conservancy
for goods or services.
\item A Conservancy Interested Person (or a family member of same) is a
director, officer, agent, partner, associate, employee, trustee, personal
representative, receiver, guardian, custodian, legal representative
or in some other way has a fiduciary duty to an entity involved in
a transaction with Conservancy. 
\item A Conservancy Interested Person (or a family member of same) is engaged
in some capacity or has a material financial interest in a business
or enterprise that competes with Conservancy or a Conservancy Project. 
\item A Conservancy Interested Person (or a family member of same) has a
material financial interest in or fiduciary duty to an entity Conservancy
has engaged in a free software license compliance effort, enforcement
effort, or related litigation on behalf of a Conservancy project. 
\item A Conservancy Interested Person (or a family member of same) has a
material financial interest in or fiduciary duty to the competitor
of an entity Conservancy has engaged in a free software license compliance
effort, enforcement effort, or related litigation on behalf of a Conservancy
project. 
\end{itemize}
Conservancy acknowledges that other situations may create the appearance
of a conflict, or present a duality of interests. All such circumstances
should be disclosed to the Board, as appropriate, and the Board shall
make a decision as to what (if any) course of action Conservancy or
relevant Conservancy Interested Persons should take so that Conservancy's
best interests are not compromised by personal interests. 


\subsection{General Policies for Conservancy Interested Persons}
\begin{itemize}
\item \textbf{No Personal Profit or Gain.} No Conservancy Interested Person
(or family member of the same), shall derive any personal profit or
gain, directly or indirectly, by reason of his or her participation
with Conservancy. 
\item \textbf{Disclosure and Abstention when Conflicted.} Each Conservancy
Interested Person shall disclose to Conservancy's Board any conflict
of interest which he or she may have in any matter pending before
Conservancy and shall refrain from participation in any decision on
such matter. 
\item \textbf{Conservancy Conflict Disclosure Form.} Every six months, each
Conservancy Interested Person shall complete a Conservancy Conflict
Disclosure form {[}FIXME: if this is posted online, then we can have
a link; if not, attach it as an Exhibit A{]} and submit it to the
Board and to Conservancy's General Counsel. 
\end{itemize}

\subsection{Conflict Resolution Procedures for Conservancy Interested Persons}
\begin{itemize}
\item \textbf{Disclosure of Conflict When Present.} Prior to any Board or
Board Committee action on a matter or transaction involving a conflict
of interest, a Conservancy Interested Person having a conflict of
interest and who is in attendance at the meeting shall disclose all
facts material to the conflict. Such disclosure shall be reflected
in the minutes of the meeting. If board members are aware that Staff
or other volunteers have a conflict of interest, relevant facts should
be disclosed by the board member or by the interested person him/herself
if invited to the Board meeting as a guest for purposes of disclosure. 
\item \textbf{Disclosure of Conflict When Absent. }A Conservancy Interested
Person who plans not to attend a meeting at which he or she has reason
to believe that the Board or Board Committee will act on a matter
in which he or she is conflicted shall disclose to the Chair of the
meeting all facts material to the conflict of interest. The Chairperson
shall report the disclosure at the meeting and the disclosure shall
be reflected in the minutes of the meeting.
\item \textbf{Participation in Discussions and Votes Regarding Conflicted
Matter.} A conflicted Conservancy Interested Person shall not participate
in or be permitted to hear the Board's or Board Committee's discussion
of the matter where he or she has a conflict of interest, except to
disclose material facts and to respond to questions. The conflicted
Conservancy Interested Person shall not attempt to exert his or her
personal influence with respect to the matter, either at or outside
the meeting. 
\item \textbf{Participation in Votes Regarding Conflicted Matter.} A conflicted
Conservancy Interested Person may not vote on the Board action with
which he or she has a conflict of interest, and shall not be present
in the meeting room (or on the conference call) when the vote is taken.
His or her ineligibility to vote shall be reflected in the minutes
of the meeting. 
\item \textbf{Conflicted Persons Cannot Establish Quorum.} A conflicted
Conservancy Interested Person shall not be determining the presence
of a quorum for purposes of a vote on the matter where he or she has
a conflict of interest. 
\item \textbf{Managing an Officer's Conflict of Interest. }If a Conservancy
Interested Person is an Officer involved in a decision, matter or
transaction in which he or she has a conflict of interest, he or she
must immediately disclose all facts material to the conflict to the
Chair of the Board (or the Chair's designee). The Board must then
approve any future decisions, negotiations, and/or other actions taken
by the Officer regarding the conflicted matter, and include the person's
disclosure of the conflict and the Board's subsequent actions in the
minutes of the next meeting. 
\item \textbf{Managing a Staff Member's Conflict of Interest.} If a Conservancy
Interested Person is a Staff Member who has been assigned duties that
involve a decision, matter or transaction in which he or she has a
conflict of interest, he or she must immediately disclose all facts
material to the conflict to the Executive Director (or the Executive
Director's designee). The Executive Director (or designee) must then
approve any future decisions, negotiations, and/or other actions taken
by the Staff Member regarding the conflicted matter, and file a written
report acknowledging the potential conflict. 
\item \textbf{Confidentiality of Conflict Disclosures.} Each Conservancy
Interested Person shall exercise care not to disclose confidential
information acquired in connection with disclosures of conflicts of
interest or potential conflicts, which might be adverse to Conservancy's
interests. 
\end{itemize}

\section{Project Leadership Committees}

PLCs are comprised of volunteers, academics, and industry professionals
that represent a Project's community and make decisions about a Project's
technical direction (``Representatives''). Conservancy understands
and expects that many Representatives exploit professional skills
relating to their Project as individuals by providing developing,
consulting, and/or training services. Nonetheless, each Representative
has a duty to act in the best interests of his or her Project when
making technical decisions about the Project.


\subsection{Defining a Conflict of Interest for a Representative}

In general, Representative should avoid making technical decisions
on matters where their personal and/or professional interests are
at odds with his or her Project's interests. In particular, the following
scenarios are to be identified as conflicts of interest:
\begin{itemize}
\item A Representative (or a family member of same) is a party to a contract,
or involved in a transaction with Conservancy for goods or services
relating to his or her Project.
\item A Representative (or a family member of same) is a director, officer,
agent, partner, associate, employee, trustee, personal representative,
receiver, guardian, custodian, legal representative, or in some way
has a fiduciary duty to an entity involved in a transaction with Conservancy
relating to his or her Project. 
\item A Representative (or a family member of same) is engaged in some capacity
or has a material financial interest in a business or enterprise that
competes with his or her Project. 
\item A Representative (or a family member of same) is the owner of copyrights
that are the subject of a Conservancy-led compliance effort, enforcement
effort, or related litigation - and the Representative (or family
member of same) has a material financial interest in or fiduciary
duty to an entity adverse to this effort. 
\end{itemize}
Conservancy acknowledges that other situations may create the appearance
of a conflict, or present a duality of interests. All such circumstances
should be disclosed to Conservancy's Executive Director and to the
PLC, as appropriate, and the PLC shall make a decision as to what
(if any) course of action the PLC or relevant Representatives should
take so that the Project's best interests are not compromised by personal
interests. 


\subsection{General Policies for Representatives}
\begin{itemize}
\item \textbf{No Compensation for Representatives.} No Representative shall
receive any salary or other substantial benefit from Conservancy as
compensation for his or her duties as a Representative. 
\item \textbf{Disclosure and Abstention when Conflicted.} Each Representative
shall disclose to his or her PLC and to Conservancy's Executive Director
any conflict of interest which he or she may have in any matter pending
before the PLC and shall refrain from participation in any decision
on such matter. 
\item \textbf{Multiple Employees from the same Employer on a PLC. }Conservancy
discourages the practice of having multiple employees of the same
employer serve on the same PLC. This practice increases the impact
of any prospective conflict of interest with the employer on the PLC,
and PLCs will have to exercise greater care to avoid the influence
of the employer's interests. If this situation is unavoidable, PLCs
are encouraged to err on the side of caution in identifying all potential
conflicts of interest relating to the employer. 
\item \textbf{Conservancy is Final Arbiter.} Each Representative acknowledges
that Conservancy is the final arbiter of any issue relating to potential
conflict.
\item \textbf{Project Conflict Disclosure Form. }Each Representative shall
complete a Project Conflict Disclosure form {[}FIXME: if this is posted
online, then we can have a link; if not, attach it as an Exhibit A{]}
and submit it to the PLC and to Conservancy's Executive Director on
an annual basis. 
\end{itemize}

\subsection{Conflict Resolution Procedures for Representatives}
\begin{itemize}
\item \textbf{Disclosure of Conflict When Present.} Prior to any PLC or
PLC sub-committee action on a matter or transaction involving a conflict
of interest, a Representative having a conflict of interest and who
is in attendance at the meeting shall disclose all facts material
to the conflict. Such disclosure shall be reflected in the minutes
of the meeting. 
\item \textbf{Disclosure of Conflict When Absent. }A Representative who
plans not to attend a meeting at which he or she has reason to believe
that the PLC or PLC sub-committee will act on a matter in which he
or she is conflicted shall disclose to the Chair of the meeting all
facts material to the conflict of interest. The Chair shall report
the disclosure at the meeting and the disclosure shall be reflected
in the minutes of the meeting.
\item \textbf{Participation in Discussions and Votes Regarding Conflicted
Matter.} A conflicted Representative shall not participate in or be
permitted to hear the PLC's or PLC sub-committee's discussion of the
matter where he or she has a conflict of interest, except to disclose
material facts and to respond to questions. The conflicted Representative
shall not attempt to exert his or her personal influence with respect
to the matter, either at or outside the meeting. 
\item \textbf{Participation in Votes Regarding Conflicted Matter.} A conflicted
Representative may not vote on the Board action with which he or she
has a conflict of interest, and shall not be present in the meeting
room (or on the conference call) when the vote is taken. His or her
ineligibility to vote shall be reflected in the minutes of the meeting. 
\item \textbf{Conflicted Persons Cannot Establish Quorum.} A conflicted
Representative shall not be determining the presence of a quorum for
purposes of a vote on the matter where he or she has a conflict of
interest. 
\end{itemize}

\subsection{Procedures for Conservancy Retaining Representative's Services}

Notwithstanding the above, Conservancy acknowledges that many Representatives
are software developers who can provide professional services useful
to advance computing and contribute to Conservancy's mission. In many
instances, a Representative will have the strongest mix of credentials,
experience, and available bandwidth to fulfill a software development
contract desired by Conservancy and/or a Project. To address those
instances, Conservancy requests Projects to follow the following procedures.
\begin{itemize}
\item \textbf{Drafting the Software Development Proposal.} PLCs must draft
a written proposal for every software development project their Project
wishes to fund. During the drafting process, if a Representative (or
family member of same), a Representative's employer and/or a fellow
employee of Representative's employer wish to be considered a candidate
to fulfill the funded software development contract, that Representative
has a conflict of interest and must recuse herself or himself from
the proposal drafting process, and abstain from any vote to approve
that proposal. All other procedures as outlined in Section 3.3 shall
still apply. The PLC must document the Representative's abstention
from the proposal drafting process in the minutes of the next PLC
meeting. 
\item \textbf{Selecting a Representative to Fulfill a Contract. }Once a
PLC has drafted and approved a development proposal, the PLC is free
to consider qualified candidates to fulfill the funded contract. If
the PLC wishes to recommend that Conservancy contract with a Representative
to carry out the work, the following criteria must be met:

\begin{itemize}
\item \textbf{Suggested Compensation.} The PLC must provide Conservancy's
Executive Director (or designee) with a suggested compensation (converted
into an hourly wage) for the software developer to be retained to
fulfill the funded contract.
\item \textbf{Independent Assessment of Credentials.} The PLC (or an unaffiliated
PLC member) must provide Conservancy's Executive Director (or designee)
with a written assessment as to why the Representative is uniquely
qualified to fulfill the funded contract.
\item \textbf{Conservancy Retains Right to Request Competitive Bids.} PLCs
acknowledge that Conservancy retains the right to ask for bids from
software developers in addition to Representative to fulfill a given
contract. Should that instance arise, Conservancy's Executive Director
(or designee) will consult with the PLC to select the candidate best
suited to fulfill the contract within the budget allotted.
\end{itemize}
\item \textbf{Conservancy holds Sole Authority to Negotiate and Execute
Contracts. }PLCs acknowledge that Conservancy holds sole authority
to negotiate and execute contracts on behalf of Member Projects. In
turn, Conservancy pledges to negotiate all contracts zealously, putting
the best interests of the affected Member Project first. To avoid
any conflicts, PLCs must not engage in any pre-negotiation with prospective
contractors - including Representatives - beyond collecting the terms
of the developer(s)' bid. 
\end{itemize}

\section{Project Community Members: Participating Corporations and Volunteers}

The work accomplished by Conservancy and its Member Projects would
not be possible without the generous donation of time, funds, and
support Project Community Members - including participating corporations,
sponsors, and volunteers alike. Community members are not traditionally
considered to be ``interested persons,'' all decision-making authority
rests with the PLCs and/or Conservancy. 
\begin{itemize}
\item \textbf{Community Members Cannot Direct Funds.} Community Members
are free to offer suggestions and engage in open dialogue with PLC,
key developers regarding a Project's technical direction. However,
each PLC and Conservancy must together maintain sole and final control
over that Project's technical direction and charitable mission. Community
Members who make financial donations do not receive any additional
control over a Project's technical direction beyond what is available
to other vocal, active, and contributing community members. \end{itemize}

\end{document}
