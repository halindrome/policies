%% LyX 2.0.1 created this file.  For more info, see http://www.lyx.org/.
%% Do not edit unless you really know what you are doing.
\documentclass[english]{article}
\usepackage[T1]{fontenc}
\usepackage[latin9]{inputenc}
\usepackage{babel}
\begin{document}

\title{Software Freedom Conservancy Travel and Reimbursable Expense Policy }

\maketitle

\section{Purpose }

This Travel and Reimbursable Expense Policy (``Policy'') applies
to all Conservancy Member Projects (``Projects'') of Software Freedom
Conservancy (``Conservancy'') and has been created to memorialize
Conservancy's reimbursement policies relating to travel and other
business expenses incurred by Conservancy staff, Project Leadership
Committee (``PLC'') members, and project volunteers while engaged
in business on behalf of, or at the behest of Conservancy and/or a
Project (``Travelers''). 

Conservancy must maintain effective control of business�-related expenses
in order to maintain its financial viability and tax exempt status.
As such, Conservancy expects persons to use good judgment and to claim
reimbursement for only those expenses that are necessary and reasonable.
Excessive expenses, including but not limited to luxury accommodations
and services unnecessary for or unrelated to the furtherance of Conservancy's
charitable mission are not eligible for reimbursement. 

Any travel expense that adheres to this Policy is considered In-Policy
and does not require special approval, so long as the trip itself
has been approved in writing by Conservancy's Executive Director or
by a Project's Leadership Committee (``PLC'') in a regular and documented
PLC vote. Conservancy and/or a PLC can limit allowable travel expenses
to an amount less than what would otherwise be considered acceptable
according to this Policy. If so, the smaller budget is the maximum
allowed expense. 


\section{Project Leadership Committee Review}

Conservancy foresees the need for periodic reasonable exceptions to
this Policy. Persons working on behalf of a specific Project seeking
an exception to this Policy must petition their PLC to obtain written
approval from Conservancy authorizing the exception. Persons working
directly on behalf of Conservancy must seeking an exception to the
Policy must obtain written approval from Conservancy authorizing the
exception. 

PLCs are responsible for creating procedures for requesting exceptions,
and submitting to Conservancy reimbursement requests associated with
their respective Projects. PLCs are also responsible for making available
a list of required response times for inquiries, including but not
limited to, the following two cases: a specific number of days to
respond to regular reimbursement requests, and a specific number of
days to respond to pre-authorization requests. 


\section{Air Travel}


\subsection{Overall Airfare Cost }

Domestic travel greater than U.S.\$750 requires Conservancy approval
prior to booking, even if all other Policy conditions have been met.
International travel greater than U.S.\$1,500 requires Conservancy
approval prior to booking, even if all other Policy conditions have
been met.


\subsection{Class of Service }

Coach Airfare is the only acceptable class for all flights (domestic
and international) unless a PLC provides a special exception and a
valid reason to Conservancy for written approval. Travelers may select
their airline of choice (e.g., for the purpose of collecting airline
miles and rewards), provided that the resulting air fare otherwise
meets the requirements of this Policy. Travelers should not book out-of-Policy
trips (and thus pay a higher fare) in order to qualify for a mileage
upgrade. 


\subsection{Advance Purchase }

Travel should be booked at least 14 days in advance; any travel booked
less than 14 days in advance requires written pre-authorization by
Conservancy. Flights beyond 365 days in advance also require written
pre-authorization by Conservancy. 


\subsection{Low Fare}

Conservancy aims to balance cost savings with convenience and considers
flights that are within U.S.\$200 of the lowest logical fare to be
within Policy. Any flights greater than U.S.\$200 over the lowest
logical fare require written pre�-authorization by Conservancy. PLCs
shall use Kayak.com as the comparison shopping site for determining
the baseline lowest fare. 


\subsection{Reasonable Flights }

Conservancy asks that travelers allow for flexibility with respect
to departure times during a desired day of travel, as well as longer
trips in order to reduce cost. However, Conservancy does consider
flights with two or more connections as unreasonable and does not
expect travelers to consider those flight options to be reasonable.


\subsection{Additional Days of Travel}

Travelers often seek to add extra days before or after an approved
trip (e.g., the weekend before a conference). A traveler may seek
approval for the expenses associated with an extended stay prior to
booking the trip, provided that the additional days are solely to
enable a traveler to conduct work within the PLC's objectives and
Conservancy's charitable mission. Travelers may seek approval to book
travel itineraries that include extra days for personal reasons, so
long as the cost of the flight meets the other requirements of this
Policy. Other expenses incurred during extra personal days beyond
air fare and transportation to and from the airport are not reimbursable. 


\subsection{Excess Baggage }

Should a team member travel on an airline that charges for a single
piece of checked baggage, such a baggage expense is eligible for reimbursement.
Team members are responsible for charges on any baggage beyond a single
piece. 


\subsection{Out-of-Policy Bookings }

All air travel not adhering to the above Policies are considered Out-of-Policy
and require written pre-authorization by Conservancy's Executive Director.


\subsection{Cancellation Fees}

Cancellation fees and other penalties incurred result of a change
of plans are reimbursable at Conservancy's discretion. In general,
Conservancy shall reimburse such fees if the traveler can submit a
valid reason for the change of plans. Acceptable reasons include Conservancy
and/or the PLC canceling or altering the trip or unexpected delays
in flight connections. In instances where these fees are incurred
without adequate explanation, Conservancy reserves the right to refuse
to reimburse the cost of the fees. 


\section{Other Reimbursable Expenses }

Conservancy will reimburse persons for Project-related expenses that
are incurred while traveling on approved Project business and/or approved
Conservancy business. Only necessary, ordinary and reasonable expenses
are eligible for reimbursement, and only those categories of expenses
listed in this document qualify. 


\subsection{Lodging }

In some cases, Conservancy or a PLC may decide to book lodging on
behalf of travelers. In this case, Conservancy�-booked lodging is
always considered In-Policy. If a traveler wants to stay elsewhere
or self-book at the same location, Conservancy will only reimburse
the team member the amount that it would otherwise have paid. Travelers
are expected to be cost-conscious and prudent when booking lodging
for approved trips. Lodging documentation submitted as part of a reimbursement
request must include a copy of the hotel invoice detailing all charges
(credit card receipts alone are unacceptable). Conservancy will not
reimburse travelers for any costs associated with an upgrade of room
accommodations. 


\subsection{Meals and Incidental Expenses}


\subsubsection{Overview}

Travelers can submit for a per diem for meals and incidental expenses
for every day of a trip devoted to PLC- and/or Conservancy-related
mission work, including the day(s) of travel itself. Maximum per diem
rates for travel within the United States shall be based on the United
States General Services Administration's Per Diem calculator (www.gsa.gov/perdiem),
using the ZIP code of the travel destination. For example, a traveler
based in Pao Alto, CA heading to a conference in Atlanta, GA (with
a layover in Dallas, TX each way) will use the ZIP code of the destination
in Atlanta to calculate the maximum per diem for the entire trip,
including days of travel. Maximum per diem rates for locations outside
of the United States shall be based on the United States Department
of State's ``Foreign Per Diem Rates by Location'' calculator's ``Meals
\& Incidental Expenses'' (M\&IE) column (http://aoprals.state.gov/web920/per\_diem.asp),
using the name of a listed city closest to the travel destination.
For example, a traveler based in Pao Alto heading to a conference
in London, England (with a layover in New York City) has a maximum
per diem of the M\&IE per diem listed London for the entire trip,
including days of travel. 

Conservancy encourages travelers to be conservative with their per
diem submissions. 

PLCs and/or Conservancy have the authority to set lower per diems
than those generated by the calculators above. In those instances,
travelers will only be able to submit for the lower per diems. 


\subsubsection{Group Meals }

For groups of travelers on an In-Policy trip, each traveler should
pay for his/her own meals, seeing as all participants will have an
opportunity to submit for separate per diems after the trip. 

For clarification purposes, this Policy does not relate to planned
group events that include meals and/or refreshments (e.g., a PLC-organized
conference that includes lunch for all attendees). Further, PLCs and/or
Conservancy retain the right to allocate a separate budget for anticipated
large group meals beyond the individual per diem limits of each traveler,
provided that they are within the PLC's technical objectives and/or
Conservancy's mission. Travelers anticipating a need to cover such
a large group meal may seek pre-approval from his/her PLC and/or Conservancy
for such expenses before the trip. 


\subsubsection{Meals For Organizational Development}

Travelers may occasionally have the need to invite third parties,
e.g., prospective donors, contributors, community members, etc., to
meals in order to further a PLC's technical direction and/or Conservancy's
mission. Conservancy recommends that travelers seek pre-approval from
their PLC and/or Conservancy for such meals. 


\subsubsection{Phone Call Charges Part of Per Diem}

Charges for personal phone calls (e.g., made from a hotel, or via
a mobile phone in international travel) are not reimbursable as an
expense separate from the allocated per diem. 


\subsection{Ground Transportation }

Ground transportation necessary as part of authorized Project trips
is considered to be a reasonable expense. Public ground transportation,
such as taxis, shuttles, buses and municipal transit, are generally
the most cost-effective options and are the standard for eligible
ground transportation reimbursements. All car rentals require pre-authorization
by the PLC or by Conservancy's Executive Director. When car rentals
have been pre�-approved, the rental of compact cars is encouraged;
mid-size vehicles are authorized when necessary (e.g., when compact�-sized
vehicles are not available or the number of passengers or volume of
baggage makes a compact vehicle impractical). 


\subsection{Rail Transportation}

Rail transportation as a means of travel for an authorized Project
trip is considered to be a reasonable expense. All rail transportation
must be in economy class in North America. 


\subsection{Use of Personal Vehicles }

When circumstances require travelers to utilize their personal vehicles
for Project purposes, he/she can be reimbursed at the current IRS
rate per mile, plus any related parking expenses and toll fees.


\section{Non-reimbursable Expenses}

Non-reimbursable expenses are identified throughout this policy. The
following items are typically non-reimbursable expenses: 
\begin{itemize}
\item Partner, spouse, and/or companion travel
\item First class travel
\item Upgrades to air travel, car rentals, or hotel rooms
\item Purchase of clothing, luggage, toiletries and other miscellaneous
personal items 
\item Supplemental travel or car rental insurance
\item Fines, penalties, or legal fees
\item Personal entertainment or recreational expenses beyond the allotted
per diem
\end{itemize}

\section{Satisfaction of IRS Requirements }

Reimbursed travel expenses are subject to examination by the Internal
Revenue Service (IRS). Travelers are responsible for retaining documentary
evidence that all expenses are strictly for Project- and/or Conservancy-related
purposes, not personal in nature, and therefore not includable as
taxable income to the traveler. Receipts are required for all expenses,
no matter the amount.


\section{Currency Exchange Rates }

Expenses incurred in foreign currencies will be converted to US dollars
based on the exchange rate on the date of incursion. The currency
exchange rate of record shall be the one posted at Oanda.com. 


\section{Approvals}

Travelers traveling on behalf of a Project must seek approvals and
submit expense reports to their PLC. PLCs are to review those expense
reports and pass them along to Conservancy's Executive Director for
final approval and reimbursement. 

Travelers traveling on behalf of Conservancy must seek approvals and
submit expense reports to Conservancy's Executive Director. Conservancy
staff must seek approvals and submit expense reports to Conservancy's
Treasurer or to another Conservancy Board Member designated by Conservancy's
Board. 


\section{Expense Reporting}

Travelers seeking reimbursement must submit an expense report to the
appropriate channel with the following information:
\begin{itemize}
\item Name of traveler 
\item Brief description of trip and trip's purpose (e.g., ``August 2011
trip to XYZ conference for ABC project, served as planning committee
member''; ``Feb. '12 FOO hackfest in Portland, OR; contributed code'')
\item Number of days traveled (with documentary evidence, e.g., conference
itinerary, etc.) and associated per diem
\item List of expenses not covered by per diem (e.g., transportation, lodging)
with substantiating receipts (or scans of receipts) 

\begin{itemize}
\item In the event that it is impractical to obtain a required receipt or
if such receipt has been inadvertently destroyed, the traveler should
furnish a written statement to that effect, as well as an explanation
of the expenditure involved
\end{itemize}
\end{itemize}
Any expense without a substantiated receipt and/or a supporting written
statement will not be reimbursed. 

Conservancy requests that all expense reports be submitted within
two weeks of travel. Expense reports filed more than 60 days after
expenses are incurred will not be reimbursed without the approval
of Conservancy's Board. 


\section{Consequences of Policy Violations}

Failure to comply with this policy may result in the denial of, or
delay in payment for, reimbursement requests. 


\section{Policy Changes }

The Conservancy reserves the right to change any terms of this Policy
from time to time. The Policy of record shall be the Policy most recently
distributed by the Conservancy. 
\end{document}
